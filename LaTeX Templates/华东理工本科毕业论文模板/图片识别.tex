\begin{document}
\mathrm{~  i n c l u d e 〈 m a t h . ~ h} 
int main()
int  n  :
 \operatorname{scanf}\left({ }^{\prime \prime} \% \mathrm{~d}^{\prime \prime}, \operatorname{sn}\right) ;
int  a[n], i, j, t 
for  (i=0 ; \quad i<n ; i++)\{ 
scanf ("\%d", \&a [i]);
for  \left(\mathbf{j}=0 ; \mathbf{j}<\mathrm{n}^{-1} ; \mathrm{j}++\right) 
for  \left(\mathbf{i}=0 ; \quad \mathbf{i}<\mathrm{n}^{-1}-\mathbf{j} ; \quad \mathbf{i}++\right) 
 \operatorname{if}(\operatorname{fabs}(\mathrm{a}[\mathrm{i}])<\operatorname{fabs}(\mathrm{a}[\mathrm{i}+1])) 
 t=a[i] ; 
 a[i]=a[i+1] ;
 a[i+1]=t ;
 \operatorname{for}\left(\mathbf{i}=0 ; \mathbf{i}<\mathrm{n}^{-1} ; \mathbf{i}++\right) 
printf ("\%d "  \mathrm{a}[\mathrm{i}]) ;
if  \left(i=n^{-1}\right) 
 \operatorname{printf}\left(" \% \mathrm{~d}^{\prime \prime}, \mathrm{a}[\mathrm{i}]\right) ;
return 0int main()
int  n, \quad \mathrm{i}=0, \mathrm{j}, \mathrm{m}[10][10] ;
 \operatorname{scanf}\left(" \% \mathrm{~d}^{\prime \prime}, \quad\right.  \&n  ) ;
for  (\mathrm{i}=0 ; \mathrm{i}<\mathrm{n} ; \mathrm{i}++)\{ 
 \mathbf{j}=0 ;
while  (\mathrm{j}<\mathrm{n}) \quad\{ 
 \operatorname{scanf}\left(" \% 1 d^{\prime \prime}, \quad \&[i][j]\right) ;
jo+;
\}
\}
for  (i=0 ; i<n ; i++) 
for  (j=0 ; j<\mathbf{j} ; \mathbf{j}++) \quad\{ 
if  (\operatorname{m}[\mathrm{i}][\mathrm{j}]==0) 
continue;
else \{
 \operatorname{printf}\left(" \mathrm{NO}^{\prime \prime}\right) ;
return 0 ;
\}
\}
\}
printf (  \left.{ }^{\prime \prime} \mathrm{YES}^{\prime \prime}\right) ;
return 0 ;
\end{document}