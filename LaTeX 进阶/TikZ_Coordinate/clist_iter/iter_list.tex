\documentclass[12pt]{article}
\usepackage[a4paper, total={6.5in, 10in}]{geometry}
\usepackage{expl3, l3sys-shell}
\usepackage{tikz}
\usetikzlibrary{intersections}


\begin{document}
\ExplSyntaxOn
% using clist \lis 
\clist_new:N \lisa
\clist_set:Nn \lisa {(1, 1); (2, 2); (3, 3)}
\clist_item:Nn \lisa{-1}

% using seq_set_split
\par
\seq_set_split:Nnn \lisb { ; } { (1, 1); (2, 2); (3, 3); (4, 4) }
% \seq_item:Nn \lisb{1}, \seq_item:Nn \lisb{-1}
\int_step_inline:nnnn {1}{1}{4}{
    \seq_item:Nn \lisb{#1}
}
% use map_inline is convenient
\par\seq_map_inline:Nn \lisb{
    % \seq_item:Nn \lisb{#1}
    #1
}
\par\seq_use:Nnnn \lisb{;}{;}{;}


\newcommand{\test}[1]{
    \seq_set_split:Nnn \lisc { ; }{#1}
    The~ First~ is:\seq_item:Nn \lisc{1},The Second~ is:\seq_item:Nn \lisc{2}. 
}
\par\test{(1, 1); (2, 2); (3, 3); (4, 4)}


\section{storage intersections}
\begin{center}
    \begin{tikzpicture}
        \draw[orange, name~ path=line] (-2, -2) -- (1.8, 2) -- (1.8, -2.2);
        \draw[green, name~ path=circle] (0, 0) circle (2);
        % get the intersections(by don't get it by return value)
        \path[name~ intersections={of=line~ and~ circle}];
        \fill[red] (intersection-1) circle (1pt); 
        \fill[blue] (intersection-2) circle (1pt); 

    % get all intersections
    % \path[name~ intersections={of=line1~ and~ line2}]; 
    % put each intersection to seq 
    \seq_set_split:Nnn \l__all_intersections_seq{;}{}
    \int_step_inline:nnnn {1}{1}{5}{
        \seq_put_right:Nn \l__all_intersections_seq{(intersection-#1)}
    }
    % show all intersections by \ShowPoint
    % \node at (0, -3) {\seq_use:Nnnn \l__all_intersections_seq{}{;}{;}};
    % \node at (0, -4) {\cs_meaning:N \l__all_intersections_seq};
    % \node at (0, -1.5) {\seq_item:Nn \l__all_intersections_seq{1}};
    \fill[blue] \seq_item:Nn \l__all_intersections_seq{2} circle(5pt);
    % \ShowPoint[#1]{\seq_use:Nnnn \l__all_intersections_seq{;}{;}{;}}
\end{tikzpicture}
\end{center}


\ExplSyntaxOff
\end{document}