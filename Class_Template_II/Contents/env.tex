\section{环境}
\subsection{正文环境}
由于本模板的简洁特性,于是相应的正文环境也进行了相应的简化,没有 
什么光彩夺目的文本环境。但是你可以把marginnote和footnote看作
一种环境

\subsubsection{Emphasis env}
\subsubsection{List Env}
\subsubsection{Recall Env}



\subsection{数学环境}
本模板定义了如下的数学环境,都是相较于简洁的,符合整个模板的灰色主题。具体每一个环境的使用方法和
效果,请参见如下的示例:

\subsubsection{definition Env}
\begin{bytes}[10]
\begin{definition}[Euler Formula]
    测试如下著名的Euler公式, Take the famous Euler's Formula As Example:
    \begin{align}
        \sum_{i=1}^{+\infty}{\int_{0}^{i}-\frac{1}{t}\mathrm{d}t} = \frac{\pi^2}{6}
    \end{align}
\end{definition}
\end{bytes}
\begin{definition}[Euler Formula]
测试如下著名的Euler公式, Take the famous Euler's Formula As Example:
\begin{align}
    \sum_{i=1}^{+\infty}{\int_{0}^{i}-\frac{1}{t}\mathrm{d}t} = \frac{\pi^2}{6}
\end{align}
\end{definition}
\marginnote{定义与定理环境可能在后期会进行合并,或单独的更改}

\subsubsection{theorem Env}
\begin{bytes}[10]
\begin{theorem}
    含有一阶导函数的泛函定义如下:
    \begin{align}
        \begin{aligned}
            \delta F
            & = F[x,U+\delta U,U'+\delta U']-F[x,U,U']  \\
            & = \left[\frac{\partial F}{\partial U}\delta U+\frac{\partial F}{\partial U'}\delta U'\right]+\frac{\partial^2F}{\partial U\partial U'}\delta U\delta U'+\frac{1}{2!}\left[\frac{\partial^2F}{\partial U^2}\delta U^2+\frac{\partial^2F}{\partial U'^2}\delta U'^2\right]+\cdots \\
            & = ...
        \end{aligned}
    \end{align}
\end{theorem}
\end{bytes}
\begin{theorem}
含有一阶导函数的泛函定义如下:
\begin{align}
    \begin{aligned}
        \delta F
        & = F[x,U+\delta U,U'+\delta U']-F[x,U,U']  \\
        & = \left[\frac{\partial F}{\partial U}\delta U+\frac{\partial F}{\partial U'}\delta U'\right]+\frac{\partial^2F}{\partial U\partial U'}\delta U\delta U'+\frac{1}{2!}\left[\frac{\partial^2F}{\partial U^2}\delta U^2+\frac{\partial^2F}{\partial U'^2}\delta U'^2\right]+\cdots \\
        & =\varepsilon\left[\frac{\partial F}{\partial U}\eta+\frac{\partial F}{\partial U'}\eta^{\prime}\right]+\frac{\varepsilon^{2}}{2}\left[\cdots\right.
    \end{aligned}
\end{align}
\end{theorem}

\subsubsection{proposition Env}
\begin{bytes}[10]
\begin{proposition}[某命题]
    这是一个命题环境, this is a proposition env
\end{proposition}
\end{bytes}
\begin{proposition}[某命题]
    这是一个命题环境, this is a proposition env
\end{proposition}
\marginnote{这部分的内容可能也会在后期有着较大的变动}

\subsubsection{corollary Env}
\begin{bytes}[10]
\begin{corollary}[某推论]
    这是一个推论环境, this is a corollary env
\end{corollary}
\end{bytes}
\begin{corollary}[某推论]
    这是一个推论环境, this is a corollary env
\end{corollary}

\subsubsection{lemma Env}
\begin{bytes}[10]
\begin{lemma}[某引理]
    这是一个引理环境, this is a lemma env
\end{lemma}
\end{bytes}
\begin{lemma}[某引理]
    这是一个引理环境, this is a lemma env
\end{lemma}

\subsection{抄录环境}
本模板提供了部分代码抄录环境 \verb|bytes, code|,抄录环境\verb|code|使用\verb|\lstset|命令进行设置,
比如下面指定编程语言的高亮选择,这里以Python举例.\footnote[2]{lislisting已经定义了几十种语言的高亮,详情请参见lislisting的2.4.1小节}

\subsubsection{Specific Language}
\begin{code}{python}
for circle_index in range(CIRCLE_NUM):
    coors_set_pure = np.array([item for item in coors_set[circle_index, :] if item is not None and point_is_in(item)])
    point_amount += coors_set_pure.size/2
    if coors_set_pure.size == 0:
        continue
    # 把有效坐标保存
    if circle_index == 0:
        all_coors = pd.DataFrame(coors_set_pure)
\end{code}

\begin{code}{C}
int main() {
    int data[] = {8, 7, 2, 1, 0, 9, 6};
    int n = sizeof(data) / sizeof(data[0]);
    printf("Original array: \n");
    for (int i = 0; i < n; i++) {
        printf("%d ", data[i]);
    }
    printf("\n");
    
    quickSort(data, 0, n - 1);
    
    printf("Sorted array: \n");
    for (int i = 0; i < n; i++) {
        printf("%d ", data[i]);
    }
    return 0;
}    
\end{code}
\marginnote[-23em]{注意:这里的语言类型C/C++需要大写,不然会报错}

\subsubsection{Just listing}
当然我们也可以不指定编程语言的类别,直接原样输出:
\begin{bytes}
\newcounter{definition}
\newenvironment{definition}[1][]{%   
        \stepcounter{definition}%
        \begin{tcolorbox}[%
            enhanced,
            arc=3pt,
            frame hidden,
            % suppressfootnotes=true,
}{\end{tcolorbox}}
\end{bytes}


\vfill
\textcolor{red}{\kaishu 关于footnote还有一个bug:如果height没有填满,那么footnote就不会到最底部}
