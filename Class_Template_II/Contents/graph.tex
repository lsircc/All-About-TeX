\section{图形绘制}
在开发本文档类 \zlatex{}时,我也同时开发了对应的绘图宏包。
往往来说,新手认为在\TeX{}中绘制图形是一件很麻烦的事情,即便是
经常用\LaTeX{}的朋友,在绘制一些简单的图形时,也会觉得比较麻烦。
但是在\TeX{}中绘图,麻烦的还不知语法的本身,更重要的是引入 \verb|tikz|
这中大型宏包之后,本身的编译速度就会变得十分的缓慢,于是本模板开率开发其他的宏包
用于调用外部函数进行绘图,减少文档的编译时长。

\zlatex{}宏包中内置了\texttt{FunctionPlot}宏包,其作用就是用来绘制各种图像,
包括普通形状绘制,2维函数绘制,3维函数绘制。自己在之前的命令基础上又重新定义了两个使用
gnuplot的绘图命令,这样可以使得编译速度进一步加快。同时也对图形设置了默认的选项,免去了
读者去翻阅官方文档的麻烦。

\subsection{基本图形绘制}
其中就包含基础的圆,三角形,直角三角形,平行四边形,梯形,平行线,
以及各种基本图形。
对应的命令调用格式如下:
\begin{bytes}
% 1.圆
\circle[center]{radius}
% 2.三角形
\triangle[center]{side}
% 3.矩形
\rectangle[center][width][angle]{height}
\end{bytes}


% \clearpage
\subsection{函数绘制}
函数的绘制类型是十分的丰富的,包括:
\marginnote{注意:这里只列出了部分图形绘制类型,完整列表请参考官方文档。对应的tikz指令也可以用户自定义}

\begin{framed}
\begin{multicols}{3}
    \begin{itemize}    
        \item 显式函数
            \begin{itemize}
                \item 二维函数
                \item 3维函数
                \item 参数方程
                \item 数据绘图
            \end{itemize}
        \item 隐式函数
        \begin{itemize}
            \item 2维函数
            \item 3维函数
            \item 参数方程
            \item 数据绘图
        \end{itemize} 
        \item 外部程序
        \begin{itemize}
            \item gnuplot 
            \item mathematica
            \item matlab
            \item python
        \end{itemize} 
\end{itemize} 
\end{multicols}
\end{framed}


\newcommand{\Note}[1]{\texttt{#1}}
至于普通的二三维的函数绘图十分简单的,你可以指定一下绘制的颜色,
或者说是colormap,定义域之类的plot parameters,尽情发挥。
下面我们主要绘制了 $y=x\sin(2x)$, $z=x*y$两个函数,效果还是可以的。

\nomargin
% \Gplot[scale]{color}{f(x)}
\begin{center}
    \Gplot[1.2]{blue, domain=-12:12}{x*sin(2*x)}
    \Gplotz[.08]{bluered}{x*y}   

    \Gplotz[.1]{cool}{x*y}
\end{center}

\subsection{参数方程绘图}
主要是在vscode中定义了下面5个绘制命令的trigger
\begin{bytes}
trigger    --> 展开式
plot2d     --> \Gplot[scale]{color}{f(x)}
plot3d     --> \Gplotz[scale]{colormap style}{f(x, y)}
polarplot  --> \polarplot[scale][plot parameters]{f(\t)}
paraplot2d --> \paraplot[scale][plot parameters]{{x(t)},{y(t)}}
paraplot3d --> \paraplotz[scale][plot parameters]{{x(t)},{y(t)}}
\end{bytes}

其实参数方程作图主要就是\Note{极坐标},\Note{二维参数方程},
\Note{三维参数方程}这三种常见的情形.
那么我们就首先绘制极坐标的图形,下面我们绘制图形对应的方程分别为:
\begin{align}
    & \rho = \frac{0.01}{\pi \theta}\\
    & \rho = \sin(\theta)
\end{align}
\margin

\marginnote[2em]{只需要注意的一个点:paraplot默认使用的是角度,而非弧度}

\begin{center}
    \polarplot[.7][red, domain=0:1440]{0.01/pi*\t}
    \polarplot[2][blue,thick,domain=0:120]{sin(\t)}
\end{center}


既然极坐标我们能够画出来了,那么接下来就是参数方程了,
\begin{center}
    \paraplot{{3*cos(t)}, {sin(t)}}
    \paraplot[.75][domain=1:2, orange, very thick]{{t}, {-2.5*(t-1.5)^2}}
\end{center}

最后就是我们的三维参数方程了,本质也是和上面的二维方程一样的,因为我们默认
$z(t) = t$,演示效果如下:下面我们绘制了螺旋线的方程 $x=\sin(t),y=3*\cos(t)(y=\cos(t)),z=t$
\begin{center}
    \paraplotz[.8]{{sin(t)}, {3*cos(t)}}
    \paraplotz[.8][blue,very thick,domain=0:1440]{{sin(t)}, {cos(t)}}
\end{center}

同理,对应的MMA模块我也归纳到了FunctionPlot宏包中,
用于调用MMA生成对应的pdf矢量图片

