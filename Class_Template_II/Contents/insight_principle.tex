\section{设计原则}
\subsection{简洁}
本模板没有花里胡哨的各种设计元素,主题颜色偏向灰色,没有很多的色彩,如果
使用者想要更加多彩设计元素可以使用\verb|ElegantLaTeX|系列模板.

简洁也就意味着本模板没有使用过多宏包,大部分命令均是通过 \LaTeX2$\varepsilon{}$ 的一些
primitive定义,甚至部分结构的达成是通过 \TeX{}的primitive完成。这样的一个好处就是,
本模板的使用者可以在此模板基础上自由的接入其他宏包,而不必考虑宏包的冲突问题。


\subsection{编译速度}
由于大部分人均是使用的windows,只要你的文章稍微多一些,速度就下去了。于是本模板充分
考虑到\LaTeX{}中各个耗时的宏包,代码块,编译引擎差异。在tikz绘图和数值计算方面均采用
外部程序完成,这样可以一定程度上增加编译速度。


\subsection{自定义}
在这一章节我会深入的介绍本模板的工作方式,以及其中定义的命令.比如我们这里对content进行了重写:
\marginnote{这里自定义命令大部分基于\LaTeX2$\varepsilon{}$}[-12em]

\begin{bytes}
%% 1.原始自定义命令
\raise\the\dimexpr-\ch@pterskip-\title@skip%
    \hbox{\rule{.03\textwidth}{.006\textheight}}\kern-1em%
\raise\the\dimexpr-\ch@pterskip-\title@skip%
    \hbox{\rule{\dimexpr.95\textwidth-1em}{.5pt}}
\raise\the\dimexpr\ch@pterskip-2em\hbox{\vspace*{\the\ch@pterskip}}%
\kern-1em\large Chapter \Roman{chapter}: #1
\end{bytes}


同时考虑到overfull hbox,改进了原始的命令,改进后的命令如下,此时我们便已经修复了原始contents中的
overfull hbox问题:
\begin{bytes}
% 使用新的chapter命令
\let\old@chapter\chapter
\renewcommand{\chapter}[1]{%
    \ifnum\value{chapter}=1\relax
        \setlength{\title@skip}{-4em}
    \else%
        \setlength{\title@skip}{0em}
    \fi
    \addtocontents{toc}{%
        \protect\mbox{}\protect\hspace*{1.7em}%
            \rule[\the\title@skip]{.03\textwidth}{.006\textheight}%
        \rule[\the\title@skip]{\dimexpr.96\textwidth-2em}{.5pt}\par%
    }
    \old@chapter{#1}
}
\end{bytes}
