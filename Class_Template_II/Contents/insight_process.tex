\section{设计流程}
本模板的设计流程大致如下:
\begin{itemize}
    \item 首先确定自己的页面布局,因为我这个模板是希望能够用于记笔记的,
        并且希望有margin,于是就选择了一个较宽的页面布局,这就是为什么和其他的模板不同
    \item 在确定自己的基本页面尺寸后便可以考虑另一个比较大的方面:页脚,页眉,本模板采用的是\verb|fancyhdr|宏包
    \item 然后是这个模板的语言,一般来说大部分人都是自用,也就没有去考虑英文,在这里我想说的是,你得有这种意识
        让你的模板能够有更多得适应面,也可以提高你使用 \TeX{}的能力
    \item 接下来是正文部分,在正文部分你首先应该考虑你需要使用的计数器(文章结构),需要chapter吗? 需要part吗?
        确定层级顺序后便可以着手设计你的目录,本模板使用的是 \verb|tocloft|宏包
    \item 接下来应该才是大部分人关心的部分,什么数学环境,代码环境,表格,图片等
\end{itemize}
