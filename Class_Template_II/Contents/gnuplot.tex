\section{~gnuplot}
鉴于TiKZ本身的运算能力,使用\verb|tikz|来绘制数据量比较大的图形时显然是不明智的,
于是我们在这里引入了\verb|gnuplot|,让 \TeX{} 在外边调用\verb|gnuplot|来绘制图形,以此加快 tex
文档的编译速度.

\subsection{gnuplot 的安装}
安装gnuplot是很简单的

\subsection{gnuplot调用}
想要使用外部程序\verb|gnuplot|, 必须要开启本模板的\verb|--shell-escape|参数,在命令行手
动编译时添加如下:
\begin{bytes}
xelatex --shell-escape <file>.tex
\end{bytes}

如果你是使用的 vscode,那么可以仿照如下的配置对你的原始 json 文件进行设置:
\begin{bytes}
{
    "name": "xelatex",
    "command": "xelatex",
    "args": [
        "-synctex=1",
        "--shell-escape",
        "-interaction=nonstopmode",
        "-file-line-error",
        "%DOC%"
    ]
},
\end{bytes}

\subsection{脚本介绍}
本模板中的使用方法是十分的简单的,我已经把\verb|gnuplot|的基础导出设置进行了设置,
并把这些文件放在模板文件夹下的\verb|Scripts|文件夹. 首先说明脚本内容:
\begin{bytes}
----- 879 21 Sep 16:13 Gplot_2d.gp
----- 774 22 Sep 12:33 Gplot_3d.gp
----- 742 21 Sep 15:59 Gplot_3d.ps1
----- 997 22 Sep 12:33 GPplot.gp
\end{bytes}

\subsection{使用方法}
本模板在此基础上,定义了如下几个命令:
\begin{itemize}
    \item \verb|\GplotzNew{fun}|: 绘制三维显示函数
    \item \verb|\GPplot{fun1, fun2, fun3}|: 绘制三维隐函数,也即三维参数方程绘制
\end{itemize}

使用语法是十分的简单的, 如下为示例:
\begin{bytes}
\GplotzNew{x+y}
\GplotzNew{x**2-y**2-10}
\end{bytes}

\begin{center}
    \GplotzNew{x+y}
    \GplotzNew{x**2-y**2-10}
\end{center}


\begin{bytes}
\GPplotz[-.75:.75]{1*cos(u)*cos(v), 2*cos(u)*sin(v), sin(u)}
\GPplotz[-4.5:4.5]{4*cos(u)*cos(v), 4*cos(u)*sin(v), 4*sin(u)}
\end{bytes}

\begin{center}
    \GPplotz[-.75:.75]{1*cos(u)*cos(v), 2*cos(u)*sin(v), sin(u)}
    \GPplotz[-4.5:4.5]{4*cos(u)*cos(v), 4*cos(u)*sin(v), 4*sin(u)}
\end{center}




