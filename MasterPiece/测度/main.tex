\documentclass[12pt]{article}
\PassOptionsToPackage{quiet}{fontspec}
\usepackage{ctex}
\usepackage{Note}
\usepackage[lite,subscriptcorrection,slantedGreek,nofontinfo]{mtpro2}

\title{线性变换与测度}
\author{04-15}
\date{}
\begin{document}
\maketitle

\section{题目}
(题源\footnote[1]{题目来源:Stein实分析P32习题8})假定 $T$ 是 $\B{R}^d$ 上的线性变换,若 $E$ 是 $\B{R}^d$ 的可测子集,
证明
\begin{enumerate}
    \item $T(E)$可测
    \item $m(T(E)) = \big|\det (T)\big|\cdot m(E)$
\end{enumerate}

\vspace*{3em}
\noindent\rule{.9\linewidth}{1pt}

一些记号:

1. $B$ 为方体 
\begin{align*}
    B = \left\{x\in \prod_{i=1}^{d}(\alpha_i, \beta_i]\;\bigg|\;\alpha_i\le \beta_i\right\}
\end{align*}

2. $A$ 为一零测集, 即 $m^*(A) = 0$



\clearpage
\section{解法1}
定义一个映射 $T:\B{R}^d \lr \B{R}^d$
\begin{Definition}[引理1]
    $E$ 可测 \equ $\exists\; \{\C{F}_n\}_{n\ge 1}$与 $A$,使
    \begin{align}
        E = \mathop{\cup}\limits_{n=1}^{\infty}\C{F}_n\bigcup A
    \end{align}
     其中 $\C{F}_n$是闭集, $A$ 是零测度集
\end{Definition}

\begin{Definition}[引理2]
    先假设以下的这个引理成立
    \begin{align}
        & A \subset B \lrr T(A) \subset T(B) \notag \\
        & T(A\cup B) = T(A) \cup T(B)
    \end{align}
\end{Definition}

\subsection{引理证明}
$E$ 可测,则可以知道 $\forall A> 0$, $\exists$ 闭集  $\C{F}$, 使得
$m^*(E-F) < \varepsilon,\; F\in E$ 。我们可以知道: $\exists \C{F}_n$
使得 $m^*(E-\C{F}_n) < \frac{\varepsilon}{2^n}$,
令  $F = \cup_{n=1}^{\infty}{\C{F}_n}$,则 $m^*(E-F) = m^*(E-\cup_{n=1}^{\infty}{\C{F}_n}) \le \varepsilon$.
记 $A = E-F$即可。

另一方面:若 $E = \cup_{n=1}^{\infty}\C{F}_n\bigcup A$,显然 $E$ 可测。
$\square$

\noindent{\bf (1)证明}

根据引理(1)可知,$\exists$ 紧集列 $\{\C{F}_n\}_{n\ge 1}$与零测集 $A$, 
使 $E = \cup_{n=1}^{\infty}{\C{F}_n}\bigcup A$,那么就有
\begin{align}
    T(E) = \cup_{n=1}^\infty{T(\C{F}_n)}\bigcup T(A)
\end{align}
此时显然可以知道 $T(E)$可测。

\noindent{\bf (2)证明}

$\forall \varepsilon > 0, \exists\;\{C_n\}_{n\ge 1} \supset E$,
根据外侧度的次可加性有:
\begin{align*}
     & m^*(E) + \frac{\varepsilon}{\bigl|\det T\bigr|} \ge \sum_{n=1}^{\infty}{m^*(C_n)} \\
\lrr & \bigl|\det T\bigr| \cdot m^*(E) + \varepsilon \ge \sum_{n=1}^{\infty}{m^*(T(C_n))} \\
\mathop{\lrr}\limits^{\varepsilon \to 0} & \bigl|\det T\bigr| \cdot m^*(E) \ge {m^*(T(E))}
\end{align*}

对于 零测集 $A$ ,则 $\forall \varepsilon > 0,\; \exists\; \{C_n\}_{n\ge 1},\; C_n \in B$,有如下限制 
\begin{align*}
    \frac{\varepsilon}{\bigl|\det T\bigr|} \ge \sum_{n=1}^{\infty}{m^*(C_n)}
    \lrr \varepsilon \ge \sum_{n=1}^{\infty}{\bigl|\det T\bigr|\cdot m^*(C_n)} 
    = \sum_{n=1}^{\infty}{m^*(T(C_n))}
     \ge m^*(T(A))
\end{align*}

使用 $T$ 的逆变换 $T^{-1}$ ,自然就可以得到反向的不等式
\begin{align*}
    \bigl|\det T^{-1}\bigr| \sum_{n=1}^{\infty}{m^*(T(C_n))} \ge \cdot m^*(E)
\end{align*} 

于是就有
\begin{align*}
    \bigl|\det T\bigr| \cdot m^*(E) = \sum_{n=1}^{\infty}{m^*(T(C_n))}
\end{align*}

\section{解法2}
\noindent{\bf (1)证明}

$E$  是一个紧集,则根据线性变换的连续性可以知道 $T(E)$ 也是一个紧集,
于是根据紧集的性质,如下的关系成立:
\begin{align*}
    E = \bigcup_{n=1}^{N}{\C{F}_n} 
      = \bigcup_{n=1}^{N}{\left(\bigcup_{k=1}^{K}{\C{F}_{n, k}}\right)} 
\end{align*}

其中的 $\C{F}_n$为紧集(因为:紧集可以使用有限个开覆盖覆盖,我们自然可以找到这样的有限个紧集覆盖)
因为 $E$ 可测,于是 $R^d$中所有可测集构成一个 $F_{\sigma}$集。 应用线性变换后有:
\begin{align*}
    T(E) & = \bigcup_{n=1}^{N}{\left(\underbrace{\bigcup_{k=1}^{K}{T\left(\C{F}_{n, k}\right)}}_{F_\sigma \mbox{集}}\right)}\\
         & = \bigcup_{n=1}^{N}\underbrace{T(\C{F}_n)}_{F_\sigma \mbox{集}}  \\
    \lrr & T(E) \mbox{可测}(\mbox {因为可数个}F_\sigma \mbox{仍然为}F_\sigma\mbox{集} )  
\end{align*}


\noindent{\bf (2)证明}

因为显然有 $\bigl|T(x) - T(x')\bigr| \le \sqrt{d}ML$,
其中的 $M$ 方体 $B$
的{\bf 直径},即 $M = \mathop{\max}\limits_{x_i, x_j \in B}|x_i-x_j|$。
所以 线性变换  $T$实际上进行了如下的操作:对 $\forall x\in B$
\begin{align*}
    x = (x_1, x_2, \cdots, x_d) 
    & \mathop{\lr}\limits^{T} x' = (\delta_1x_1, \delta_2x_2, \cdots, \delta_dx_d) \\
    & \subset x'' = (\sqrt{d}Mx_1, \sqrt{d}Mx_2, \cdots, \sqrt{d}Mx_d)
\end{align*}
于是根据第(7)题类似的论证可以知道有 :
\begin{align*}
    \bigl|\det T\bigr| m^*(E) \ge m^*(T(E))
\end{align*}

仿照第一种证明方法,我们进行相应的逆变换即可得到反向的不等式,于是即有:
\begin{align*}
    \bigl|\det T\bigr| \cdot m^*(E) = m^*(T(E))
\end{align*}



\end{document}