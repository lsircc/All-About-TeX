\documentclass[12pt]{article}
\PassOptionsToPackage{quiet}{fontspec}
\usepackage[a4paper, total={6.5in, 10in}]{geometry}
\usepackage{amsmath}
\usepackage{amsfonts}
\usepackage{enumerate}
\usepackage{tikz}
\usepackage{ctex}
\usepackage{pgfplots}
\pgfplotsset{compat=1.17}
\usepackage{xcolor}
\usepackage{float}
\usepackage{graphicx}
\usepackage[lite,subscriptcorrection,slantedGreek,nofontinfo]{mtpro2}

\title{Gauss迭代}
\author{zongpingD}
\date{\today}
\begin{document}
\maketitle

\section{对角占优矩阵}

直接使用一个矩阵的行列式性质即可
\begin{align}
    C\equiv\lambda(D-L)-U=
    \begin{bmatrix}\lambda a_{11}&a_{12}&\cdots&a_{1n}\\ 
        \lambda a_{2}&\lambda a_{22}&\cdots&a_{2n}\\ 
        \vdots&\vdots&&\vdots\\ 
        \lambda a_{n1}&\lambda a_n&\cdots&\lambda a_n
    \end{bmatrix}
\end{align}

\subsection{对角占优定理}
对角占优定理需叙述如下:
如果 $A = (a_{ij})_{n\times n}$ 为严格对角占优矩阵或者是
不可约弱对角占优矩阵,则 $A$为非奇异矩阵

有如下的证明过程
\begin{align*}
    \mid c_{ij}\mid
    & = \mid\lambda a_in\mid>\mid\lambda\mid(\sum\limits_{j=1}^{j+1}\mid a_i\mid+\sum\limits_{j=i+1}^n\mid a_i\mid) \\
    & \geq \sum_{j=1}^{i-1}\mid\lambda a_{ij}\mid+\sum_{j=i+1}^{n}\mid a_{ij}\mid\\
    & =\sum_{j=1}^{n}\mid c_{ij}\mid i=1,2,\cdots,n.
\end{align*}





\end{document}