\documentclass[12pt]{article}
\PassOptionsToPackage{quiet}{fontspec}
\usepackage[a4paper, total={6.5in, 10in}]{geometry}
\usepackage{amsmath}
\usepackage{amsfonts}
\usepackage{bm}
\usepackage{enumerate}
\usepackage{tikz}
\usepackage{ctex}
\usepackage{pgfplots}
\pgfplotsset{compat=1.17}
\usepackage{xcolor}
\usepackage{float}
\usepackage{framed}
\usepackage{multicol}
\usepackage{booktabs}
\usepackage{graphicx}


% ////////////////////////////////////////////////////////////////////
% //                          _ooOoo_                               //
% //                         o8888888o                              //
% //                         88" . "88                              //
% //                         (| ^_^ |)                              //
% //                         O\  =  /O                              //
% //                      ____/`---'\____                           //
% //                    .'  \\|     |//  `.                         //
% //                   /  \\|||  :  |||//  \                        //
% //                  /  _||||| -:- |||||-  \                       //
% //                  |   | \\\  -  /// |   |                       //
% //                  | \_|  ''\---/''  |   |                       //
% //                  \  .-\__  `-`  ___/-. /                       //
% //                ___`. .'  /--.--\  `. . ___                     //
% //              ."" '<  `.___\_<|>_/___.'  >'"".                  //
% //            | | :  `- \`.;`\ _ /`;.`/ - ` : | |                 //
% //            \  \ `-.   \_ __\ /__ _/   .-` /  /                 //
% //      ========`-.____`-.___\_____/___.-`____.-'========         //
% //                           `=---='                              //
% //      ^^^^^^^^^^^^^^^^^^^^^^^^^^^^^^^^^^^^^^^^^^^^^^^^^^        //
% //            佛祖保佑       永不宕机     永无BUG                    //
% ////////////////////////////////////////////////////////////////////


\title{\LaTeX{}从入门到入土}
\author{Eureka}
\date{\today}






\begin{document}
\begin{titlepage}
    \maketitle
\end{titlepage}
\clearpage


\begin{center}
    \tableofcontents    
\end{center}
\thispagestyle{empty}
\setcounter{page}{0}
\clearpage


\section{环境配置}
\LaTeX{} 的环境配置对于新手来说是比较的劝退的,所以我在下面推荐两种使用\LaTeX{}的途径:
\begin{itemize}
    \item 使用在线平台
    \item 本地自行配置
\end{itemize}

新手建议使用在线平台,有着许多的优秀的在线平台。


\subsection{发行版}

\subsubsection{TeXLive}

\subsubsection{MiKTeX}

\subsubsection{MacTeX}

\subsection{在线环境}
如:overleaf, TeXPage等
\subsubsection{OverLeaf}

\subsubsection{Slager}

\subsubsection{TeXPage}



\subsection{本地配置}
\subsubsection{文档阅读器}
包括最常用的pdf阅读器, ps阅读器, dvi阅读器, svg查看器 
\subsubsection{正反检索}
\subsubsection{编译链}




\subsection{前端 -- 编辑器选择}
\subsubsection{TeXWorks}

\subsubsection{TeXStudio}

\subsubsection{TeXMaker}

\subsubsection{VsCode}

\subsubsection{Sublime}

\subsubsection{Vim or Emacs}




\section{第一篇文档}
我们当然得从{\ttfamily Hello World!}出发,看看我们是怎么打印出Hello, World的。



\section{文档结构介绍}
\subsection{导言区}

\subsection{正文}


\section{\TeX{}家族}
\subsection{历史概论}


\subsection{编译器与格式}


\subsection{输出格式}


\section{常用宏包}



\section{图片}
\subsection{外部图片}

\subsection{外部程序}

\subsection{手动绘制}





\section{公式}



\section{表格}




\section{字体}



\section{代码}


\section{文学编程}

\subsection{宏编程}

\subsection{当前 -- \LaTeX{} 2$\varepsilon$}


\subsection{未来 -- \LaTeX{} 3 }




\section{结语}























\end{document}